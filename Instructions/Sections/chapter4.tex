\section{数学环境示例}

\paragraph{说明}
\begin{enumerate}
	\item 提供了\textit{定理,定义,引理,推论,命题,问题,练习,例,注}环境
	\item 每种定理打*号的是完整版,可以添加定理的名字和标签,以下是\texttt{template.tex}中定义定理环境的代码:
	\begin{verbatim}
		\newenvironment{thm*}[2]{\begin{thm}[#1]\label{#2}}{\end{thm}}
	\end{verbatim}
\end{enumerate}

使用方式:
\begin{verbatim}
\begin{defi*}{劳仑衣普桑}{def1}
	劳仑衣普桑,认至将指点效则机,最你更枝。想极整月正进好志次回总般,段然取向使张规军证回,世市总李率英茄持伴。用阶千样响领交出,器程办管据家元写,名其直金团。化达书据始价算每百青,金低给天济办作照明,取路豆学丽适市确。如提单各样备再成农各政,设头律走克美技说没,体交才路此在杠。响育油命转处他住有,一须通给对非交矿今该,花象更面据压来。与花断第然调,很处己队音,程承明邮。常系单要外史按机速引也书,个此少管品务美直管战,子大标蠢主盯写族般本。农现离门亲事以响规,局观先示从开示,动和导便命复机李,办队呆等需杯。见何细线名必子适取米制近,内信时型系节新候节好当我,队农否志杏空适花。又我具料划每地,对算由那基高放,育天孝。派则指细流金义月无采列,走压看计和眼提问接,作半极水红素支花。果都济素各半走,意红接器长标,等杏近乱共。层题提万任号,信来查段格,农张雨。省着素科程建持色被什,所界走置派农难取眼,并细杆至志本。
\end{defi*}

\begin{pf}
	定理引用示例:由~\ref{def1},显然成立。
\end{pf}

\end{verbatim}

给出:

\begin{defi*}{劳仑衣普桑}{def1}
劳仑衣普桑,认至将指点效则机,最你更枝。想极整月正进好志次回总般,段然取向使张规军证回,世市总李率英茄持伴。用阶千样响领交出,器程办管据家元写,名其直金团。化达书据始价算每百青,金低给天济办作照明,取路豆学丽适市确。如提单各样备再成农各政,设头律走克美技说没,体交才路此在杠。响育油命转处他住有,一须通给对非交矿今该,花象更面据压来。与花断第然调,很处己队音,程承明邮。常系单要外史按机速引也书,个此少管品务美直管战,子大标蠢主盯写族般本。农现离门亲事以响规,局观先示从开示,动和导便命复机李,办队呆等需杯。见何细线名必子适取米制近,内信时型系节新候节好当我,队农否志杏空适花。又我具料划每地,对算由那基高放,育天孝。派则指细流金义月无采列,走压看计和眼提问接,作半极水红素支花。果都济素各半走,意红接器长标,等杏近乱共。层题提万任号,信来查段格,农张雨。省着素科程建持色被什,所界走置派农难取眼,并细杆至志本。
\end{defi*}

\begin{pf}
	定理引用示例:由~\ref{def1},显然成立。
\end{pf}

\vspace{1em}

下面类似,请自行查看\texttt{chapter4.tex}中的源代码:

\begin{thm*}{Lorem ipsum}{}
Lorem ipsum dolor sit er elit lamet, consectetaur cillium adipisicing pecu, sed do eiusmod tempor incididunt ut labore et dolore magna aliqua. Ut enim ad minim veniam, quis nostrud exercitation ullamco laboris nisi ut aliquip ex ea commodo consequat. Duis aute irure dolor in reprehenderit in voluptate velit esse cillum dolore eu fugiat nulla pariatur. Excepteur sint occaecat cupidatat non proident, sunt in culpa qui officia deserunt mollit anim id est laborum. Nam liber te conscient to factor tum poen legum odioque civiuda.
\end{thm*}

\begin{lem}
	
Lorem ipsum dolor sit er elit lamet, consectetaur cillium adipisicing pecu, sed do eiusmod tempor incididunt ut labore et dolore magna aliqua. Ut enim ad minim veniam, quis nostrud exercitation ullamco laboris nisi ut aliquip ex ea commodo consequat. Duis aute irure dolor in reprehenderit in voluptate velit esse cillum dolore eu fugiat nulla pariatur. Excepteur sint occaecat cupidatat non proident, sunt in culpa qui officia deserunt mollit anim id est laborum. Nam liber te conscient to factor tum poen legum odioque civiuda.
\end{lem}

\begin{cor}
在$l^1$中弱收敛和强收敛等价。
\end{cor}

\begin{pf}
	It's trivial.
\end{pf}

\begin{prop}
在$l^1$中弱收敛和强收敛等价。
\end{prop}

\begin{rmk}
	太显然了。
\end{rmk}

\begin{pf}
	显然!
\end{pf}

\begin{eg}
在$l^1$中弱收敛和强收敛等价。
\end{eg}

\begin{prob*}{开映射定理}{}
	对$X$, $Y$为$B$空间,$T \in \mathcal{L}(X, Y)$ 为满射,则为开映射.
\end{prob*}

\begin{ex}
	证明:在$l^1$中弱收敛和强收敛等价。
\end{ex}

\newpage

\section{数学宏(newcommand)}

\paragraph{说明}
由于 \texttt{mathbb} \texttt{mathcal} 等输入较为麻烦,所以采用宏替换的方式有效减少输入量。

示例:
\begin{verbatim}
	\[ \mul{j=1}{n} y_j \]
	\[ \ipo \n f(\bs{x}) \d \bs{x} \]
	\[ \E(X) = 0, \var(X) = 1, X_1 \cdots X_n \iid X. \]
	\[ \norm{\add{i=1}{n} \lambda_i f_i}_{L^2(\Om)} \]
	\[ \rd \ \rn \]
\end{verbatim}

给出:

\[ \mul{j=1}{n} y_j \]
\[ \ipo \n f(\bs{x}) \d \bs{x} \]
\[ \E(X) = 0, \var(X) = 1, X_1 \cdots X_n \iid X. \]
\[ \norm{\add{i=1}{n} \lambda_i f_i}_{L^2(\Om)} \]
\[ \rd \ \rn \]

以下是 \texttt{template.tex} 中的源代码:

\begin{verbatim}
% 注意会使原有的 \P 无法使用
\newcommand{\R}{\mathbb{R}}
\newcommand{\C}{\mathbb{C}}  
\newcommand{\Z}{\mathbb{Z}}
\newcommand{\N}{\mathbb{N}}
\newcommand{\Q}{\mathbb{Q}}
\renewcommand{\P}{\mathbb{P}}
\newcommand{\E}{\mathbb{E}}
\newcommand{\p}{\partial}
\newcommand{\n}{\nabla}
\newcommand{\D}{\Delta}

\newcommand{\rd}{\mathbb{R}^d}
\newcommand{\rn}{\mathbb{R}^n}

\newcommand{\add}[2]{\sum_{#1}^{#2}}
\newcommand{\mul}[2]{\prod_{#1}^{#2}}

% 注意会使原有的 \d{o} 功能无法使用
\renewcommand{\d}{\,\mathrm{d}}

% 代替了\bf \cal,这两个是过时的命令
\newcommand{\bb}[1]{\mathbb{#1}}
\renewcommand{\cal}[1]{\mathcal{#1}}
\renewcommand{\bf}[1]{\mathbf{#1}}
\newcommand{\bs}[1]{\boldsymbol{#1}}

\newcommand{\ir}{\int_{\R}}
\newcommand{\ird}{\int_{\R^d}}
\newcommand{\irn}{\int_{\R^n}}
\newcommand{\io}{\int_{\Omega}}
\newcommand{\ipo}{\int_{\partial\Omega}}

\newcommand{\po}{\partial\Omega}
\newcommand{\Om}{\Omega}

% text 'in', \in already exist
\newcommand{\tin}{\text{in }}
\newcommand{\on}{\text{on }}
\newcommand{\as}{\text{as }}
\newcommand{\for}{\text{for }}
\newcommand{\dist}{\text{dist }}
\newcommand{\vol}{\text{vol}\,}
\newcommand{\iid}{\text{ i.i.d. }}
\newcommand{\var}{\text{Var}}

\newcommand{\eps}{\varepsilon}
\newcommand{\vphi}{\varphi}

% 角括号 < >
\providecommand{\angle}[1]{\langle #1 \rangle }
% 范数 || ||
\providecommand{\norm}[1]{\left\| #1 \right\| }
% 绝对值 | |
\providecommand{\abs}[1]{\left| #1 \right| }
% 小括号 ( )
\providecommand{\pare}[1]{\left( #1 \right)}
\end{verbatim}

\newpage
\section{插图示例}

\paragraph{说明}
这里对latex图片插入功能做了打包,封装成了下面的函数:
\begin{verbatim}
% fig 函数:简约版
% 参数结构:
% #1: 图片路径(必选)
% #2: 图片宽度(必选)
% #3: 图片标题(必选)
\NewDocumentCommand{\fig}{m m m}{%
	\begin{figure}[htbp]  % 使用可选参数#4作为位置参数(默认htbp)
		\centering
		% 插入图片:必选参数#1=路径,#2=宽度
		\includegraphics[width=#2\textwidth]{#1}
		% 标题:可选参数#3,非空时才显示
		\caption{#3}
	\end{figure}
}

% figu 函数:完整版
% 额外参数:
% #4: figure位置参数(可选,默认htbp)
% #5: 图片标签(可选,默认空)
\NewDocumentCommand{\figu}{m m m O{htbp} O{}}{%
	\begin{figure}[#4]  % 使用可选参数#4作为位置参数(默认htbp)
		\centering
		% 插入图片:必选参数#1=路径,#2=宽度
		\includegraphics[width=#2\textwidth]{#1}
		% 标题:可选参数#3,非空时才显示
		\IfValueT{#3}{\caption{#3}}
		% 标签:可选参数#5,非空时才生成(且仅当有标题时才加标签)
		\IfValueT{#5}{%
			\IfValueT{#3}{\label{fig:#5}}% 避免无标题却有标签的无效情况
		}
	\end{figure}
}

% minifig 函数:图片并排环境的简约版
% 参数:
% #1: 左侧图片路径
% #2: 右侧图片路径
% #3: 左侧minipage宽度
% #4: 右侧minipage宽度
% #5: 左侧图片标题
% #6: 右侧图片标题

\NewDocumentCommand{\minifig}{m m m m m m}{
	\begin{figure}[htbp]
		\centering
		\begin{minipage}[t]{#3\textwidth}  % #3: 左侧minipage宽度
			\centering
			\includegraphics[width = 0.83\linewidth]{#1}  % #1: 左侧图片路径
			\caption{#5}  % #5: 左侧图片标题
		\end{minipage}
		\begin{minipage}[t]{#4\textwidth}  % #4: 右侧minipage宽度
			\centering
			\includegraphics[width = 0.83\linewidth]{#2}  % #6: 右侧图片路径
			\caption{#6}  % #5: 右侧图片标题
		\end{minipage}
	\end{figure}
}

% minifigure 环境:图片并排环境完整版
% 额外参数:
% #7: figure位置参数(默认htbp)
% #8: 左侧图片标签(默认空)
% #9: 右侧图片标签(默认空)

\NewDocumentCommand{\minifigure}{m m m m m m O{htbp} O{} O{}}{
	\begin{figure}[#7]  % #7: figure位置参数(默认htbp)
		\centering
		\begin{minipage}[t]{#3\textwidth}  % #3: 左侧minipage宽度
			\centering
			\includegraphics{#1}  % #1: 左侧图片路径
			\caption{#5}  % #5: 左侧图片标题
			\IfValueT{#8}{\label{fig:#8}}  % #8: 左侧标签(非空才生成)
		\end{minipage}
		\begin{minipage}[t]{#4\textwidth}  % #4: 右侧minipage宽度
			\centering
			\includegraphics{#2}  % #6: 右侧图片路径
			\caption{#6}  % #5: 右侧图片标题
			\IfValueT{#9}{\label{fig:#9}}  % #9: 右侧标签(非空才生成)
		\end{minipage}
	\end{figure}
}
\end{verbatim}

下面是示例:
\begin{verbatim}
\fig{ustcblue}{0.8}{中国科学技术大学校徽}
\minifig{ustcblue}{ustcblue}{0.4}{0.4}{中国科学技术大学校徽1}{中国科学技术大学校徽2}
\end{verbatim}

给出:

\fig{ustcblue}{0.8}{中国科学技术大学校徽}
\minifig{ustcblue}{ustcblue}{0.4}{0.4}{中国科学技术大学校徽1}{中国科学技术大学校徽2}



\newpage
% If you only want Biber for a single document, it is possible by addding a so-called "magic comment".
% !TeX TXS-program:bibliography = txs:///biber

% 最新版本documentclass
\documentclass[UTF8,a4paper,12pt]{ctexart}

% 引入必要的包
\usepackage{amsmath, amssymb, amsfonts,amsthm}
\usepackage{graphicx}
\usepackage{geometry}
\usepackage{fancyhdr}
\usepackage{titlesec}
\usepackage{hyperref}
\usepackage{xcolor}
\usepackage{listings}
\usepackage{booktabs}
\usepackage{caption}
\usepackage{enumitem}
\usepackage{eso-pic}
\usepackage{fontawesome5}
\usepackage{tocloft}
\usepackage{algorithm}      % 浮动体环境
\usepackage{algpseudocode}  % 伪代码风格
\usepackage{stmaryrd}
\usepackage{xparse}  % 支持灵活的参数定义,如可选参数)
\usepackage{etoolbox}
\usepackage[toc,page]{appendix}  % toc: 将附录加入目录;page: 每个附录单独一页(可选)

% 文献引用
% \usepackage[utf8]{inputenc}
% \usepackage[backend=biber,style=numeric,citestyle=numeric]{biblatex}

% 这个只有在overleaf中可以biber代替bibertex成为默认选项
% \addbibresource{Sections/refs.bib}  % 引入 refs.bib 文件
% \hypersetup{colorlinks=true, linkcolor=black, urlcolor=black, citecolor=black}

% 设置图片目录
\graphicspath{{Figures/}}

% 设置颜色常量
\definecolor{greenblue}{RGB}{0,160,160} % 蓝绿色
\definecolor{darkgreen}{rgb}{0.0, 0.5, 0.0}  % 墨绿色
\definecolor{darkblue}{RGB}{0,76,153} % 深蓝色
\definecolor{orange}{RGB}{255,130,0} % 橙色
\definecolor{purple}{RGB}{139,0,139}

\definecolor{t1}{RGB}{65,113,199} % theme color 1
\definecolor{t2}{RGB}{255,130,0} % theme color 2

% 设置页面尺寸和边距
\geometry{
	left=2.5cm,
	right=2.5cm,
	top=2.5cm,
	bottom=2.5cm,
	headsep=10pt,
	footskip=30pt
}

% 调整页眉高度(避免警告)
\setlength{\headheight}{18.04524pt}
\addtolength{\topmargin}{-0.0pt} % 微调顶部位置

% 目录居中
\renewcommand{\contentsname}{\makebox[\linewidth][c]{\color{t1}\bfseries\Large 目录}}

% 修改目录的字体颜色为黑色
\hypersetup{
	colorlinks=true,
	linkcolor=black,  % 将目录超链接设置为黑色
	urlcolor=magenta,
	citecolor=black,
	pdfborder={0 0 0}
}
\renewcommand{\cfttoctitlefont}{\Large\bfseries} % 目录标题也加大(可选)

% 可选:设置目录条目字体为 \large(比默认大一号)
%\renewcommand{\cftsecfont}{\large}
%\renewcommand{\cftsecpagefont}{\large}
%\renewcommand{\cftsubsecfont}{\large}
%\renewcommand{\cftsubsecpagefont}{\large}
%\renewcommand{\cftsubsubsecfont}{\large}
%\renewcommand{\cftsubsubsecpagefont}{\large}
%\renewcommand{\cftaftertoctitle}{\hfill}


% 自定义标题样式
\titleformat{\section}
  {\normalfont\centering\Large\bfseries\color{t1}}  % 一级标题
  {\thesection}{1em}{}

\titleformat{\subsection}
  {\normalfont\large\bfseries\color{t1}} % 二级标题
  {\thesubsection}{1em}{}

\titleformat{\subsubsection}
  {\normalfont\normalsize\bfseries\color{t1}} % 三级标题
  {\thesubsubsection}{1em}{}

\titleformat{\paragraph}
  {\normalfont\normalsize\bfseries\color{t1}} % 四级标题
  {\theparagraph}{1em}{}

% 页眉页脚设置
\pagestyle{fancy}
\fancyhf{}
\fancyhead[L]{\textcolor{t1}{\leftmark}}
\fancyhead[R]{\textcolor{t1}{\thepage}}

% 设置页眉线颜色和粗细
\renewcommand{\headrule}{\color{t1}\hrule width\textwidth height 0.7pt \vskip 1mm}

% 右下角个人信息
%\fancyfoot[R]{\href{https://github.com/jaceywei}{\Jaceywei}}

% 代码高亮设置
\lstset{
    basicstyle=\ttfamily\footnotesize,
    keywordstyle=\color{blue},
    commentstyle=\color{gray},
    stringstyle=\color{red},
    frame=single,
    breaklines=true
}


% 定义itemize样式,将点点设置为t1
\setlist[itemize]{label=\textcolor{t1}{\textbullet}}

% 定义表的样式
\captionsetup[table]{
	name = {表},
	labelfont = {color=t1, bf}
}

% 定义图的样式
\captionsetup[figure]{
	name = {图},
	labelformat = simple,
	labelfont = {color=t1, bf}
}

% 定义算法样式
\DeclareCaptionLabelSeparator{t1on}{\textbf{\textcolor{t1}{: }}}

% 让代码中关键字变成主题颜色
% 保存原始 \textbf
%\let\originaltextbf\textbf
% 定义蓝色加粗(用原始 \textbf)
%\newcommand{\themebf}[1]{\originaltextbf{\textcolor{t1}{#1}}}
% 在 algorithmic 中启用
%\AtBeginEnvironment{algorithmic}{\let\textbf\themebf}

\captionsetup[algorithm]{
	name = {算法},
	labelformat = simple,
	labelfont = {color=t1, bf},
	labelsep = t1on  % 在编号后加冒号和空格
}

% 自定义 caption 格式:直接使用 \thelstlisting
\DeclareCaptionFormat{lstcaption}{%
	\textbf{\textcolor{t1}{代码~\thelstlisting}}%
	\textbf{\textcolor{t1}{:}}%
	\textcolor{black}{~#1#3} %#3 是 caption 内容
}
\captionsetup[lstlisting]{format=lstcaption, labelformat=empty}

% 自定义 代码 样式
\lstset{
	basicstyle=\ttfamily,
	frame=single,                     % 添加边框
	framerule=0.4pt,                  % 边框粗细
	rulecolor=\color{t1},           % 边框颜色为蓝色
	captionpos=b,                     % 标题放在下方
	numbers=none,                     % 去掉左边的灰色数字(行号)
	keywordstyle=\color{blue},
	commentstyle=\color{green},
	stringstyle=\color{red},
	showstringspaces=false,
	breaklines=true,
	postbreak=\mbox{\textcolor{red}{$\hookrightarrow$}},
	escapeinside={\%*}{*)}
}

% 修改标题名称为“代码”,并加粗变蓝
\renewcommand{\lstlistingname}{\textbf{\textcolor{t1}{代码}}}

% 定义enumerate样式
\setlist[enumerate]{label=\textcolor{t1}{\textbf{\arabic*.}}}

% fig 函数:简约版
% 参数结构:
% #1: 图片路径(必选)
% #2: 图片宽度(必选)
% #3: 图片标题(必选)
\NewDocumentCommand{\fig}{m m m}{%
	\begin{figure}[htbp]  % 使用可选参数#4作为位置参数(默认htbp)
		\centering
		% 插入图片:必选参数#1=路径,#2=宽度
		\includegraphics[width=#2\textwidth]{#1}
		% 标题:可选参数#3,非空时才显示
		\caption{#3}
	\end{figure}
}

% figu 函数:完整版
% 额外参数:
% #4: figure位置参数(可选,默认htbp)
% #5: 图片标签(可选,默认空)
\NewDocumentCommand{\figu}{m m m O{htbp} O{}}{%
	\begin{figure}[#4]  % 使用可选参数#4作为位置参数(默认htbp)
		\centering
		% 插入图片:必选参数#1=路径,#2=宽度
		\includegraphics[width=#2\textwidth]{#1}
		% 标题:可选参数#3,非空时才显示
		\IfValueT{#3}{\caption{#3}}
		% 标签:可选参数#5,非空时才生成(且仅当有标题时才加标签)
		\IfValueT{#5}{%
			\IfValueT{#3}{\label{fig:#5}}% 避免无标题却有标签的无效情况
		}
	\end{figure}
}

% minifig 函数:图片并排环境的简约版
% 参数:
% #1: 左侧图片路径
% #2: 右侧图片路径
% #3: 左侧minipage宽度
% #4: 右侧minipage宽度
% #5: 左侧图片标题
% #6: 右侧图片标题

\NewDocumentCommand{\minifig}{m m m m m m}{
	\begin{figure}[htbp]
		\centering
		\begin{minipage}[t]{#3\textwidth}  % #3: 左侧minipage宽度
			\centering
			\includegraphics[width = 0.83\linewidth]{#1}  % #1: 左侧图片路径
			\caption{#5}  % #5: 左侧图片标题
		\end{minipage}
		\begin{minipage}[t]{#4\textwidth}  % #4: 右侧minipage宽度
			\centering
			\includegraphics[width = 0.83\linewidth]{#2}  % #6: 右侧图片路径
			\caption{#6}  % #5: 右侧图片标题
		\end{minipage}
	\end{figure}
}

% minifigure 环境:图片并排环境完整版
% 额外参数:
% #7: figure位置参数(默认htbp)
% #8: 左侧图片标签(默认空)
% #9: 右侧图片标签(默认空)

\NewDocumentCommand{\minifigure}{m m m m m m O{htbp} O{} O{}}{
	\begin{figure}[#7]  % #7: figure位置参数(默认htbp)
		\centering
		\begin{minipage}[t]{#3\textwidth}  % #3: 左侧minipage宽度
			\centering
			\includegraphics{#1}  % #1: 左侧图片路径
			\caption{#5}  % #5: 左侧图片标题
			\IfValueT{#8}{\label{fig:#8}}  % #8: 左侧标签(非空才生成)
		\end{minipage}
		\begin{minipage}[t]{#4\textwidth}  % #4: 右侧minipage宽度
			\centering
			\includegraphics{#2}  % #6: 右侧图片路径
			\caption{#6}  % #5: 右侧图片标题
			\IfValueT{#9}{\label{fig:#9}}  % #9: 右侧标签(非空才生成)
		\end{minipage}
	\end{figure}
}

% 注意会使原有的 \P 无法使用
\newcommand{\R}{\mathbb{R}}
\newcommand{\C}{\mathbb{C}}  
\newcommand{\Z}{\mathbb{Z}}
\newcommand{\N}{\mathbb{N}}
\newcommand{\Q}{\mathbb{Q}}
\renewcommand{\P}{\mathbb{P}}
\newcommand{\E}{\mathbb{E}}
\newcommand{\p}{\partial}
\newcommand{\n}{\nabla}
\newcommand{\D}{\Delta}

\newcommand{\rd}{\mathbb{R}^d}
\newcommand{\rn}{\mathbb{R}^n}

\newcommand{\add}[2]{\sum_{#1}^{#2}}
\newcommand{\mul}[2]{\prod_{#1}^{#2}}

% 注意会使原有的 \d{o} 功能无法使用
\renewcommand{\d}{\,\mathrm{d}}

% 代替了\bf \cal,这两个是过时的命令
\newcommand{\bb}[1]{\mathbb{#1}}
\renewcommand{\cal}[1]{\mathcal{#1}}
\renewcommand{\bf}[1]{\mathbf{#1}}
\newcommand{\bs}[1]{\boldsymbol{#1}}

\newcommand{\ir}{\int_{\R}}
\newcommand{\ird}{\int_{\R^d}}
\newcommand{\irn}{\int_{\R^n}}
\newcommand{\io}{\int_{\Omega}}
\newcommand{\ipo}{\int_{\partial\Omega}}

\newcommand{\po}{\partial\Omega}
\newcommand{\om}{\Omega}

% text 'in', \in already exist
\newcommand{\tin}{\text{in }}
\newcommand{\on}{\text{on }}
\newcommand{\as}{\text{as }}
\newcommand{\for}{\text{for }}
\newcommand{\dist}{\text{dist }}
\newcommand{\vol}{\text{vol}\,}
\newcommand{\iid}{\text{ i.i.d. }}
\newcommand{\var}{\text{Var}}

\newcommand{\eps}{\varepsilon}
\newcommand{\vphi}{\varphi}

% 跳跃算子 [[ ]]
\newcommand{\jump}[1]{\left\llbracket #1 \right\rrbracket }
% 角括号 < >
\providecommand{\angle}[1]{\langle #1 \rangle }
% 范数 || ||
\providecommand{\norm}[1]{\left\| #1 \right\| }
% 绝对值 | |
\providecommand{\abs}[1]{\left| #1 \right| }
% 小括号 ( )
\providecommand{\pare}[1]{\left( #1 \right)}

% before upper={\setlength{\parindent}{0em}}, 用于删除所有缩进
% 如需在盒子的段首空两格,可删除before upper={\setlength{\parindent}{0em}},并在每个盒子中加入如下选项:
%	before upper={%
%	\everypar{\hspace*{2em}\everypar{}}% 每段开头加 2em 空白
%	\noindent\ignorespaces
%},

% 定义共享计数器(基于 section)
\newcounter{sharedthm}[section]
\renewcommand{\thesharedthm}{\thesection.\arabic{sharedthm}}

% 简约样式

% 证明环境 color = orange
\newenvironment{pf}{
	\par\noindent\textcolor{t2}{\textbf{Proof}}\quad
	\ignorespaces
}{
	\qed\par
}

% Remark环境
\newenvironment{rmk}{
	\par\noindent\textcolor{t2}{\textbf{注}}\quad
	\ignorespaces
}

\newtheoremstyle{colored}%
{}{}% 间距
{\itshape}{}% 正文斜体
{\color{t1}\bfseries}{}% 标题:t1 + 加粗
{ }% 分隔符
{}% 后续

\theoremstyle{colored}

\newtheorem{thm}{定理}[section]   % 主计数器
\newtheorem{defi}[thm]{定义}       % 共享计数器
\newtheorem{lem}[thm]{引理}
\newtheorem{cor}[thm]{推论}
\newtheorem{prop}[thm]{命题}
\newtheorem{prob}{问题}[section]
\newtheorem{ex}{练习}[section]
\newtheorem{eg}[thm]{例}

% ========== 封装带 {标题}{标签} 的星号版本 ==========
\newenvironment{thm*}[2]{\begin{thm}[#1]\label{#2}}{\end{thm}}
\newenvironment{defi*}[2]{\begin{defi}[#1]\label{#2}}{\end{defi}}
\newenvironment{lem*}[2]{\begin{lem}[#1]\label{#2}}{\end{lem}}
\newenvironment{cor*}[2]{\begin{cor}[#1]\label{#2}}{\end{cor}}
\newenvironment{prob*}[2]{\begin{prob}[#1]\label{#2}}{\end{prob}}
\newenvironment{prop*}[2]{\begin{prop}[#1]\label{#2}}{\end{prop}}
\newenvironment{ex*}[2]{\begin{ex}[#1]\label{#2}}{\end{ex}}
\newenvironment{eg*}[2]{\begin{eg}[#1]\label{#2}}{\end{eg}}
